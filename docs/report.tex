\documentclass[a4paper,12pt]{report}

\usepackage[utf8]{inputenc}
\usepackage{graphicx}
\usepackage{listings}
\usepackage{hyperref}

\title{Simulatore Distance Vector Routing\\Report per il Progetto di\\Reti di Telecomunicazioni A.A. 2024/2025}
\author{Marco Marrelli}
\date{Dicembre 2024}

\begin{document}
\maketitle

\renewcommand{\contentsname}{Indice}
\tableofcontents{}

\section{Introduzione}

\section{Requisiti}
\begin{itemize}
    \item Python 3.6 (o superiore)
    \item Libreria Tkinter (Framework per la parte Grafica)
    \item Librerie Standard di Python, come:
    \begin{itemize}
        \item typing (per i Type Hints)
        \item math (per Calcoli e Costanti Matematiche)
        \item random (per l'aspetto della Randomicità)
    \end{itemize}
\end{itemize}

\section{Struttura della Codebase}
\subsection{Componenti per la Grafica}
\subsection{Componenti per la Logica}

\section{Implementazione della Grafica}
\subsection{Visualizzazione Grafica della Rete}
\subsection{Manipolazione della Rete}

\section{Implementazione della Logica}
\subsection{L'Algoritmo (Distance Vector Routing)}

\section{Conclusioni}

\end{document}